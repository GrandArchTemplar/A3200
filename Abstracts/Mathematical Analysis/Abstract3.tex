\documentclass[a4paper, 12pt, titlepage, fleqn]{article}
\usepackage[cm]{fullpage}
\usepackage[T2A]{fontenc}
\usepackage[utf8]{inputenc}
\usepackage[russian]{babel}
\usepackage{times}
\usepackage{color}
\usepackage{xcolor}
\usepackage[pdftex]{graphicx}
\usepackage{indentfirst}
\usepackage{listings}
\usepackage{amsfonts}

\newtheorem{Def}{Определение}[section]
\newtheorem{Th}{Теорема}[section]
\newtheorem{Proof}{Доказательство}[section]

\newcommand{\Real}{\mathbb{R}}
\newcommand{\Int}{\mathbb{Z}}
\newcommand{\Nat}{\mathbb{N}}
\newcommand{\Rat}{\mathbb{Q}}
\newcommand{\Com}{\mathbb{C}}
\newcommand{\Irr}{\mathbb{I}}
\newcommand{\T}{\textbf}
\newcommand{\D}{\partial}
\newcommand{\Where}{\T{, где }}
\newcommand{\Indexes}{\T{, }\forall i \in (1, \dots, n)}
\begin{document}
	\begin{Large}
		\begin{center}
			Дифференцируемость ФНП
		\end{center}
	\end{Large}
	\section{Частная производная}
		\subsection{Определение частной производной}
			$\vec{x^0} = \left(x_1^0, \dots, x_n^0\right)$\\
			\begin{Def}
				Рассмотрим $f\left(x_1, \dots, x_n \right) $, тогда 
				\[
					\Delta_i f\left(\vec{x^0}\right) = f \left(\vec{x^0} + \Delta\vec{x_{i}^0}\right) - f \left(\vec{x^0}\right) \Where \Delta\vec{x_{i}^0} = \{0, \dots, \Delta x_i, \dots, 0\}	
				\]
				\[
					\frac{\D f}{\D x_i} = \lim_{\Delta x_i \to 0} \frac{\Delta_i f \left(\vec{x^0}\right)}{\Delta x_i}, \: \forall i \in \left(1, \dots, n\right)
				\]
			\end{Def}
		\subsection{Определение дифференцируемости в точке}
			\begin{Def}
				$f \left(\vec{x}\right)$ дифференцируема в $\vec{x^0}$, если \[
					\exists A_i,\: \: \forall i \in \left(1, \dots, n\right) \: \colon \: \Delta f \left(\vec{x^0}\right) = <\!\!\vec{A}, \vec{\Delta x}\!\!>, \|\vec{\Delta x}\| \to 0, \Where \Delta f \left(\vec{x^0}\right) = f \left(\vec{x^0} + \Delta\vec{x}\right) - f \left(x^0\right) 
				\]
			\end{Def}
	\section{Необходимое условие дифференцируемости}
		\begin{Th}
			Если функция дифференцируема в $\vec{x^0}$, то $\exists \forall \frac{\D f}{\D x_i}\left(\vec{x^0}\right) \Indexes$
		\end{Th}
		\begin{Proof}
			Будет позже
		\end{Proof}
	\section{Достаточное условие дифференцируемости}
		\begin{Th}
			Если $\exists \forall \frac{\D f}{\D x_i} \Indexes$ и они непрерывны в $\vec{x^0}$, то $f \left(\vec{x}\right)$ дифференцируема в $\vec{x^0}$
		\end{Th}
		\begin{Proof}
			Будет позже
		\end{Proof}
\end{document}