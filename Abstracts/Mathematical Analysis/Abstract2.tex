\documentclass[a4paper, 12pt, titlepage, fleqn]{article}
\usepackage[cm]{fullpage}
\usepackage[T2A]{fontenc}
\usepackage[utf8]{inputenc}
\usepackage[russian]{babel}
\usepackage{times}
\usepackage{color}
\usepackage{xcolor}
\usepackage[pdftex]{graphicx}
\usepackage{indentfirst}
\usepackage{listings}
\usepackage{amsfonts}

\newtheorem{Def}{Определение}[section]
\newtheorem{Th}{Теорема}[section]
\newtheorem{Proof}{Доказательство}[section]
\newcommand{\Real}{\mathbb{R}}
\newcommand{\Int}{\mathbb{Z}}
\newcommand{\Nat}{\mathbb{N}}
\newcommand{\Rat}{\mathbb{Q}}
\newcommand{\Com}{\mathbb{C}}
\newcommand{\Irr}{\mathbb{I}}
\newcommand{\T}{\textbf}

\begin{document}
	\begin{titlepage}
	\end{titlepage}
	\[
		f(x) \: x \in D \subset \Real
	\]
	\[
		x^0 \T{ -- предельная точка } D
	\]
	\section{Определение по Коши}
		\[
			A = \lim_{x \to x^0} f(x) \leftrightarrow \forall \varepsilon > 0 ,\: \exists \delta > 0 ,\: \forall x \in D \cap \dot{U_\delta} (x^0) \to f(x) \in U_\varepsilon(A)
		\]
	\section{Определение по Гейне}
		$x^{(k)}$ -- последовательность
		\[
			A = \lim_{x \to x^0} f(x) \leftrightarrow \forall x^{(k)} \: \colon\: (x^{(k)} \ne 0) \T{ и } x^{(k)} \to x^0, \: n\to\infty
		\]
	\section{Предел на множестве}
	\[
		A = \lim_{x \to x^0, \: x \in M} f(x) \leftrightarrow \forall \varepsilon > 0, \: \exists \delta > 0, \: \forall x \in D \cap \dot{U_\delta} (x^0) \cap M \to f(x) \in U_\varepsilon(A)
	\]
	Если $\exists \lim_{x \to x^0}$, то $\forall M \:\colon\: \lim_{x \to x^0, \: x \in M} = \lim_{x \to x^0} f(x)$
	
	\section{Повторный предел}
		\begin{Def}
			Пусть $f(x, y)$ определена в $(x_0 - \delta, x_0 + \delta)\times(y_0 - \delta, y_0 + \delta)$ и $\forall x \in (x_0 - \delta, x_0 + \delta) \: \exists \lim_{y \to y_0} f(x, y) = g(x)$, тогда $\exists\lim_{x \to x^0} g(x)$, называемый повторным пределом
		\end{Def}
	\section{Непрерывность}
		\begin{Def}
			$f(x)$ называется непрерывной в $x^0 \in \Real^n$, если $\lim_{x\to x^0} f(x) = f(x^0)$
		\end{Def}
		\begin{Def}
			Функция называется непрерывной на множестве $M$ в $x^0 \in M \subset \Real^n$, если $\lim_{x\to x^0} f(x) \ f(x^0)$
		\end{Def}
		\begin{Th}[о непрерывности сложной функции]
			Пусть $y_i = \varphi_i(x), i \in (1, \dots, n)$ определены в $U(x^0)$ и непрерывны в $x^0 \in \Real^m$. $f(y)$ определена в $U(y^0), y^0 = \varphi(x^0)$, непрерывна в $y^0$\\
			Тогда существует и непрерывна в $x^0, \: \phi(x^0) = f(\varphi(x))$
		\end{Th}
		\begin{Proof}
			
			\[
				\forall \varepsilon > 0, \: \exists \sigma > 0, \: \forall y \in \dot{U_\sigma}(y^0) \:\colon\: |f(y) - f(y^0)| < \varepsilon \T{ //т.к. f -- непрерывна}
			\]
			\[
				\T{т. к. } \forall \varphi_i \T{ -- непрерына по } \sigma, \: \exists \delta_i \forall x \in \dot{U_{\delta_i}}(x^0) \: \colon \: (y_i(x) - y_i(x^0)) < \frac{\sigma}{\sqrt{n}}, i \in (1, \dots, n) \to \rho(y, y^0) < \sigma 
			\]
			\[
				\delta = min(\delta_i), i \in (1, \dots, n)
			\]
			\[
				\forall x \in \dots{U_\delta}(x^0) \to \rho(y, y^0) < \sigma \to |f(y) - f(y^0)| < \varepsilon
			\]
		\end{Proof}
\end{document}