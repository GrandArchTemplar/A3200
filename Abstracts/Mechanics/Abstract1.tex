\documentclass[a4paper, 12pt, titlepage, fleqn]{article}
\usepackage[cm]{fullpage}
\usepackage[T2A]{fontenc}
\usepackage[utf8]{inputenc}
\usepackage[russian]{babel}
\usepackage{times}
\usepackage{color}
\usepackage{xcolor}
\usepackage[pdftex]{graphicx}
\usepackage{indentfirst}
\usepackage{listings}
\usepackage{amsfonts}

\newtheorem{Def}{Определение}[section]

\newcommand{\Real}{\mathbb{R}}
\newcommand{\Int}{\mathbb{Z}}
\newcommand{\Nat}{\mathbb{N}}
\newcommand{\Rat}{\mathbb{Q}}
\newcommand{\Com}{\mathbb{C}}
\newcommand{\Irr}{\mathbb{I}}
\newcommand{\T}{\textbf}
\newcommand{\D}{\partial}
\newcommand{\Where}{\T{, где }}
\newcommand{\Indexes}{\T{, }\forall i \in (1, \dots, n)}

\begin{document}
	\begin{titlepage}
	\end{titlepage}
	\section{Обобщённые координаты} 
		\[
			\vec{r} \textbf{ -- радиус-вектор}
		\] 
		\[
			\vec{v} = \dot{\vec{r}} = \frac{d \vec{r}}{dt} \T{ -- скорость}
		\]
		\[
			\vec{a} = \dot{\vec{v}} = \ddot{\vec{r}} = \frac{d^2 \vec{r}}{dt^2} \T{ -- ускорение}
		\]
		\[
			\vec{r} = (x, y, z)  \T{ -- координатная форма для радиус-вектора}
		\]
		\[
			\vec{v} = (\dot{x}, \dot{y}, \dot{z}) \T{ -- координатная форма для скорости}
		\]
		\[
			\vec{a} = (\ddot{x}, \ddot{y}, \ddot{z}) \T{ -- координатная форма для ускорения}
		\]
		Координаты для системы точек:
		\[
			\vec{r_1} = (x_1, y_1, z_1) 
		\]
		\[
			\vec{r_2} = (x_2, y_2, z_2)
		\]
		\[
			\dots
		\]
		\[
			\vec{r_n} = (x_n, y_n, z_n)
		\]
		Таким образом для задания системы из $n$ точек необходимо $3n$ координат. Тогда почему бы не перейти от строго порядка к нестрогому и задавать систему просто $3n$ координат:
		\[
			(x_1, y_1, z_1, x_2, y_2, z_2, \dots x_n, y_n, z_n) \to (q_1, q_2, \dots, q_{3n})
		\]
		\[
			q = (q_1, q_2, \dots q_{3n}) \T{ -- физики ленивые. Очень.}
		\] 
		\[
			\dot{q} = (\dot{q_1}, \dot{q_2}, \dots, \dot{q_{3n}})
		\]
		\[
			\ddot{q} = (\ddot{q_1}, \ddot{q_2}, \dots, \ddot{q_{3n}})
		\]
		\[
			\ddot{q} = f(q, \dot{q}, t) \T{ -- закон мира(оно работает, попытка избежать этого закона провальна)}
		\]
	\section{Принцип наименьшего действия}
		\subsection{Формулировка}
			\[
				S = \int_{t_1}^{t_2} L(q, \dot{q}, t)dt \T{ , где } S \T{ -- действие}
			\]
			Механическая система двигается так, чтобы до $S$ было минимальным
		\subsection{Объяснение необходимости}
			(По Ландау) Пусть $A$ и $B$ -- начальная и конечная точки движения с $q_1$, $t_1$ и $q_2$, $t_2$, тогда существуют какие-то возможные траектории движения $S, S', \dots, S^{(n)}$
			\[
				\bar{S} = min(S, S', \dots, S^{(n)}) \T{ где } \bar{S} \T{ является идеальным действием для данной системы}
			\]
			(По Фейману) Пусть каждый путь из $A$ в $B$ определяется не $S$, а $e^{i\frac{S}{\hbar}}$. ($\hbar$ -- постоянная Планка)\\
			Тогда $\rho = |\sum e^{i\frac{S}{\hbar}}|^2$\\
			
			$e^{i\frac{S}{\hbar}} = \cos(\frac{S}{\hbar}) + i\sin(\frac{S}{\hbar})$\\
			
			Таким образом, каждая из наших функций при больших $S$ при сложении гасят друг друга.\\
			
			Рассмотрим функцию $y = x^2$: заметим, что при достаточно больших $x$ $\Delta y \sim \Delta x$, а при малых $x$ $\Delta y \sim (\Delta x)^2$.\\
			
			Тогда $S$ влияющие на $\rho$ это такие $S$, что $|S - S_min| < \varepsilon$, где величина $\varepsilon$ показывает уровень квантовости мира.
			
		\subsection{Вариационное исчисление}
		\[
			y(x) \: \colon\: I = \int_{x_1}^{x_2} F(y(x), y'(x), x)dx
		\] 
		Представим функциональное одномерное пространство такое, что $\forall \varepsilon\: \exists y(x)$ и $\exists \bar{y}(x)$ для $\varepsilon = 0$\\
		\[
			y(x) = \bar{y}(x) + \psi(x) \cdot \varepsilon
		\]
		$\varepsilon$ -- искомое отклонение
		\[
			F(y(x)) = \int_{x_1}^{x_2} F(\bar{y}(x) + \psi(x) \cdot \varepsilon, \bar{y}'(x) + \psi'(x) \cdot \varepsilon, x)dx
		\]
		\[
			\frac{dI}{d\varepsilon} = \int_{x_1}^{x_2}(\frac{\partial F}{\partial y}\psi(x) + \frac{\partial F}{\partial y'}\psi'(x))dx
		\]
		$x_1$ и $x_2$ -- начальная и конечная точки\\ 
		$y(x_1) = y_1$ \\
		$y(x_2) = y_2$\\
		$\psi(x_1) = \psi(x_2) = 0$
		\[
			\int_{x_1}^{x_2} \underbrace{\frac{\partial F}{\partial y'}}_u \underbrace{\psi'(x)dx}_{dv} = \underbrace{\frac{\partial F}{\partial y}\psi(x)|_{x_1}^{x_2}}_a - \int_{x_1}^{x^2} \psi(x)\frac{dF}{dx}dx = \T{//} \int u\cdot dv = u \cdot v - \int du \cdot v 
		\]
		\[
			\psi|_{x_1}^{x_2} = 0 \to a = 0 
		\]
		\[
			= \int_{x^1}^{x_2} \left(\frac{\partial F}{\D y} \psi - \psi \frac{d}{dx} \frac{\D F}{\D y'} \right)dx = 
		\]
		\[
			= \int_{x_1}^{x_2} 
		\]
	\section{Принцип относительности Галилея}
	\section{Функция Лангранжа свободной материальной точки}
\end{document}