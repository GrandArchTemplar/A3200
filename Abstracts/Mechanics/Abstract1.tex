\documentclass[a4paper, 12pt, titlepage, fleqn]{article}
\usepackage[cm]{fullpage}
\usepackage[T2A]{fontenc}
\usepackage[utf8]{inputenc}
\usepackage[russian]{babel}
\usepackage{times}
\usepackage{color}
\usepackage{xcolor}
\usepackage[pdftex]{graphicx}
\usepackage{indentfirst}
\usepackage{listings}
\usepackage{amsfonts}

\newtheorem{Def}{Определение}[section]

\newcommand{\Real}{\mathbb{R}}
\newcommand{\Int}{\mathbb{Z}}
\newcommand{\Nat}{\mathbb{N}}
\newcommand{\Rat}{\mathbb{Q}}
\newcommand{\Com}{\mathbb{C}}
\newcommand{\Irr}{\mathbb{I}}
\newcommand{\T}{\textbf}
\begin{document}
	\begin{titlepage}
	\end{titlepage}
	\section{Обобщённые координаты} 
		\[
			\vec{r} \textbf{ -- радиус-вектор}
		\] 
		\[
			\vec{v} = \dot{\vec{r}} = \frac{d \vec{r}}{dt} \T{ -- скорость}
		\]
		\[
			\vec{a} = \dot{\vec{v}} = \ddot{\vec{r}} = \frac{d^2 \vec{r}}{dt^2} \T{ -- ускорение}
		\]
		\[
			\vec{r} = (x, y, z)  \T{ -- координатная форма для радиус-вектора}
		\]
		\[
			\vec{v} = (\dot{x}, \dot{y}, \dot{z}) \T{ -- координатная форма для скорости}
		\]
		\[
			\vec{a} = (\ddot{x}, \ddot{y}, \ddot{z}) \T{ -- координатная форма для ускорения}
		\]
		Координаты для системы точек:
		\[
			\vec{r_1} = (x_1, y_1, z_1) 
		\]
		\[
			\vec{r_2} = (x_2, y_2, z_2)
		\]
		\[
			\dots
		\]
		\[
			\vec{r_n} = (x_n, y_n, z_n)
		\]
		Таким образом для задания системы из $n$ точек необходимо $3n$ координат. Тогда почему бы не перейти от строго порядка к нестрогому и задавать систему просто $3n$ координат:
		\[
			(x_1, y_1, z_1, x_2, y_2, z_2, \dots x_n, y_n, z_n) \to (q_1, q_2, \dots, q_{3n})
		\]
		\[
			q = (q_1, q_2, \dots q_{3n}) \T{ -- физики ленивые. Очень.}
		\] 
		\[
			\dot{q} = (\dot{q_1}, \dot{q_2}, \dots, \dot{q_{3n}})
		\]
		\[
			\ddot{q} = (\ddot{q_1}, \ddot{q_2}, \dots, \ddot{q_{3n}})
		\]
		\[
			\ddot{q} = f(q, \dot{q}, t) \T{ -- закон мира(оно работает, попытка избежать этого закона провальна)}
		\]
	\section{Принцип наименьшего действия}
		\subsection{Формулировка}
			\[
				S = \int_{t_1}^{t_2} L(q, \dot{q}, t)dt \T{ , где } S \T{ -- действие}
			\]
			Механическая система двигается так, чтобы до $S$ было минимальным
		\subsection{Объяснение необходимости}
			(По Ландау) Пусть $A$ и $B$ -- начальная и конечная точки движения с $q_1$, $t_1$ и $q_2$, $t_2$, тогда существуют какие-то возможные траектории движения $S, S', \dots, S^{(n)}$
			\[
				\bar{S} = min(S, S', \dots, S^{(n)}) \T{ где } \bar{S} \T{ является идеальным действием для данной системы}
			\]
			(По Фейману) Пусть каждый путь из $A$ в $B$ определяется не $S$, а $e^{i\frac{S}{den}}$
	\section{Принцип относительности Галилея}
	\section{Функция Лангранжа свободной материальной точки}
\end{document}