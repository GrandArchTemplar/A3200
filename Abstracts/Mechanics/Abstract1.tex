\documentclass[a4paper, 12pt, titlepage, fleqn]{article}
\usepackage[cm]{fullpage}
\usepackage[T2A]{fontenc}
\usepackage[utf8]{inputenc}
\usepackage[russian]{babel}
\usepackage{times}
\usepackage{color}
\usepackage{xcolor}
\usepackage[pdftex]{graphicx}
\usepackage{indentfirst}
\usepackage{listings}
\usepackage{amsfonts}

\newtheorem{Def}{Определение}[section]
\newtheorem{Proof}{Доказательство}
\newtheorem{Th}{Теорема}
\newcommand{\Real}{\mathbb{R}}
\newcommand{\Int}{\mathbb{Z}}
\newcommand{\Nat}{\mathbb{N}}
\newcommand{\Rat}{\mathbb{Q}}
\newcommand{\Com}{\mathbb{C}}
\newcommand{\Irr}{\mathbb{I}}
\newcommand{\T}{\textbf}
\newcommand{\D}{\partial}
\newcommand{\Where}{\T{, где }}
\newcommand{\Indexes}{\T{, }\forall i \in (1, \dots, n)}

\begin{document}
	\begin{titlepage}
	\end{titlepage}
	\section{Обобщённые координаты} 
		\[
			\vec{r} \textbf{ -- радиус-вектор}
		\] 
		\[
			\vec{v} = \dot{\vec{r}} = \frac{d \vec{r}}{dt} \T{ -- скорость}
		\]
		\[
			\vec{a} = \dot{\vec{v}} = \ddot{\vec{r}} = \frac{d^2 \vec{r}}{dt^2} \T{ -- ускорение}
		\]
		\[
			\vec{r} = (x, y, z)  \T{ -- координатная форма для радиус-вектора}
		\]
		\[
			\vec{v} = (\dot{x}, \dot{y}, \dot{z}) \T{ -- координатная форма для скорости}
		\]
		\[
			\vec{a} = (\ddot{x}, \ddot{y}, \ddot{z}) \T{ -- координатная форма для ускорения}
		\]
		Координаты для системы точек:
		\[
			\vec{r_1} = (x_1, y_1, z_1) 
		\]
		\[
			\vec{r_2} = (x_2, y_2, z_2)
		\]
		\[
			\dots
		\]
		\[
			\vec{r_n} = (x_n, y_n, z_n)
		\]
		Таким образом для задания системы из $n$ точек необходимо $3n$ координат. Тогда почему бы не перейти от строго порядка к нестрогому и задавать систему просто $3n$ координат:
		\[
			(x_1, y_1, z_1, x_2, y_2, z_2, \dots x_n, y_n, z_n) \to (q_1, q_2, \dots, q_{3n})
		\]
		\[
			q = (q_1, q_2, \dots q_{3n}) \T{ -- физики ленивые. Очень.}
		\] 
		\[
			\dot{q} = (\dot{q_1}, \dot{q_2}, \dots, \dot{q_{3n}})
		\]
		\[
			\ddot{q} = (\ddot{q_1}, \ddot{q_2}, \dots, \ddot{q_{3n}})
		\]
		\[
			\ddot{q} = f(q, \dot{q}, t) \T{ -- закон мира(оно работает, попытка избежать этого закона провальна)}
		\]
	\section{Принцип наименьшего действия}
		\subsection{Формулировка}
			\[
				S = \int\limits_{t_1}^{t_2} L(q, \dot{q}, t)dt \T{ , где } S \T{ -- действие}
			\]
			Механическая система двигается так, чтобы до $S$ было минимальным
		\subsection{Объяснение необходимости}
			(По Ландау) Пусть $A$ и $B$ -- начальная и конечная точки движения с $q_1$, $t_1$ и $q_2$, $t_2$, тогда существуют какие-то возможные траектории движения $S, S', \dots, S^{(n)}$
			\[
				\bar{S} = min(S, S', \dots, S^{(n)}) \T{ где } \bar{S} \T{ является идеальным действием для данной системы}
			\]
			(По Фейману) Пусть каждый путь из $A$ в $B$ определяется не $S$, а $e^{i\frac{S}{\hbar}}$. ($\hbar$ -- постоянная Планка)\\
			Тогда $\rho = |\sum e^{i\frac{S}{\hbar}}|^2$\\
			
			$e^{i\frac{S}{\hbar}} = \cos(\frac{S}{\hbar}) + i\sin(\frac{S}{\hbar})$\\
			
			Таким образом, каждая из наших функций при больших $S$ при сложении гасят друг друга.\\
			
			Рассмотрим функцию $y = x^2$: заметим, что при достаточно больших $x$ $\Delta y \sim \Delta x$, а при малых $x$ $\Delta y \sim (\Delta x)^2$.\\
			
			Тогда $S$ влияющие на $\rho$ это такие $S$, что $|S - S_min| < \varepsilon$, где величина $\varepsilon$ показывает уровень квантовости мира.
			
		\subsection{Вариационное исчисление}
			\[
				y(x) \: \colon\: I = \int\limits_{x_1}^{x_2} F(y(x), y'(x), x)dx
			\] 
			Представим функциональное одномерное пространство такое, что $\forall \varepsilon\: \exists y(x)$ и $\exists \bar{y}(x)$ для $\varepsilon = 0$\\
			\[
				y(x) = \bar{y}(x) + \psi(x) \cdot \varepsilon
			\]
			$\varepsilon$ -- искомое отклонение
			\[
				F(y(x)) = \int\limits_{x_1}^{x_2} F(\bar{y}(x) + \psi(x) \cdot \varepsilon, \bar{y}'(x) + \psi'(x) \cdot \varepsilon, x)dx
			\]
			\[
				\frac{dI}{d\varepsilon} = \int\limits_{x_1}^{x_2}(\frac{\partial F}{\partial y}\psi(x) + \frac{\partial F}{\partial y'}\psi'(x))dx
			\]
			$x_1$ и $x_2$ -- начальная и конечная точки\\ 
			$y(x_1) = y_1$ \\
			$y(x_2) = y_2$\\
			$\psi(x_1) = \psi(x_2) = 0$
			\[
				\int\limits_{x_1}^{x_2} \underbrace{\frac{\partial F}{\partial y'}}_u \underbrace{\psi'(x)dx}_{dv} = \underbrace{\frac{\partial F}{\partial y}\psi(x)|_{x_1}^{x_2}}_a - \int\limits_{x_1}^{x^2} \psi(x)\frac{dF}{dx}dx = \T{//} \int u\cdot dv = u \cdot v - \int du \cdot v 
			\]
			\[
				\psi|_{x_1}^{x_2} = 0 \to a = 0 
			\]
			\[
				= \int\limits_{x^1}^{x_2} \left(\frac{\partial F}{\D y} \psi - \psi \frac{d}{dx} \frac{\D F}{\D y'} \right)dx = 
			\]
			\[
				= \int\limits_{x_1}^{x_2} \left( \left(\frac{\D F}{\D y} - \frac{d}{dx} \frac{\D F}{\D y'}\right) \psi \right) dx  = 0
			\]
			\[
				\to \int\limits_{x_1}^{x_2} \left(G\left(x\right) \psi \left(x\right) \right) dx = 0 \to G \equiv 0 \T{ -- основная лемма вариационного исчисления}
			\]
			\begin{Th}[Основная лемма вариационного исчисления]
				Пусть
				\[
					\int\limits_{x_1}^{x_2} \left(G\left(x\right) \psi \left(x\right) \right) dx = 0 \Where \psi(x) \: \forall f \: \colon \: \Real^n \to \Real
				\]
				, тогда
				\[
					G \equiv 0
				\] 
			\end{Th}
			\begin{Proof}
				Известно, что $\psi(x_1) = \psi(x_2) = 0$.\\
				Тогда зададим $\psi(x)$ так, что $\forall x \in \left((x1, x2) / (x_* - \delta, x_* + \delta)\right) = 0$\\	
				Получаем, что 
				\[
					\int\limits_{x_1}^{x_2} \left(G(x) \psi(x)\right) dx = \int\limits_{x_* - \delta}^{x_* + \delta} \left(G(x) \psi(x)\right) dx =
				\]
				По теореме о среднем $\exists x_{**} \in (x_* - \delta, x_* + \delta)$ такое, что
				\[
					= G(x_**) \int\limits_{x_* - \delta}^{x_* + \delta} \psi(x)dx				
				\]
				Следовательно при $\delta \to 0 ,\: x_{**} = x_*$ и $G(x_*) = 0$\\
				Это верно для любого $x \in (x_1, x_2)$
			\end{Proof}
		\subsection{Уравнения Эйлера и Лагранжа}
			\[
				I = \int\limits_{x_1}^{x_2} F\left(y(x), y'(x), x) \right)dx
			\]
			\[
				\frac{d}{dx}\frac{\D F}{\D y'} = \frac{\D F}{\D y} \T{ -- уравнение Эйлера}
			\]
			\[
				S = \int\limits_{x_1}^{x_2} L\left(q, \dot{q}, t\right)dt
			\]
			\[
				\frac{d}{dt}\frac{\D L}{\D \dot{q_i}} = \frac{\D L}{\D q_i} \Indexes \T{ -- уравнение Лагранжа(основное в лагранжевой механике)}
			\]
			\[
				\bar{q}(t), \: q = \bar{q} + \delta q 
			\]
			\[
				\delta S = S\left(\bar{q} + \delta q\right) - S(\bar{q}) = \int\limits_{t_1}^{t_2} \left(\frac{\D L \delta q}{\D q} + \frac{\D L \delta \dot{q}}{\D \dot{q}}\right) dt
			\]
			\[
				S(\bar{q} + \delta q) - S(\bar{q}) = \int\limits_{t_1}^{t_2} L(\bar{q} + \delta q, \dot{\bar{q}} + \delta \dot{q}, t)dt - \int\limits_{t_1}^{t_2} L(\bar{q}, \dot{\bar{q}}, t)dt =
			\]
			\[
				= \int \limits_{t_1}^{t_2} \left(L(\bar{q} + \delta q, \dot{\bar{q}} + \delta \dot{q}, t) - L(\bar{q}, \dot{\bar{q}}, t)\right)dt = 
			\]
			\[
				= \int \limits_{t_1}^{t_2} \left[L(\bar{q}, \dot{\bar{q}}, t) + \frac{\D L}{\D q} \delta q + \frac{\D L}{\D \dot{q}} \delta\dot{q} - L(\bar{q}, \dot{\bar{q}}, t)\right]dt = \T{ // по разложению в ряд Тейлора}
			\]
			\[
				= \int \limits_{t_1}^{t_2} \left(\frac{\D L}{\D q}\delta q + \frac{\D L}{\delta\dot{q}}\delta\dot{q}\right)dt = \int \limits_{t_1}^{t_2} \left(\frac{\D L}{\D q} - \frac{d}{dt}\frac{\D L}{\D \dot{q}} \right)dt + \underbrace{\frac{\D L}{\D \dot{q}}\delta q |_{t_1}^{t_2}}_{=0}
			\]
		\subsection{Разложение в ряд Тейлора ФНП}
			\[
				f(x) : \Real^n \to \Real
			\]
			\[
				f(x) = \frac{1}{1!}\sum\limits_{i = 1}^{n}\frac{\D f}{\D x_i}\delta x_i + \frac{1}{2!}\sum\limits_{i = 1}^{n}\sum\limits_{j = 1}^{n}\frac{\D^2 f}{\D x_j \D x_i}\delta x_i\delta x_j + \cdots
			\]
		\subsection{Свойства L}
			\subsubsection{Неопределённость L}
				\[
					\int\limits_{t_1}^{t_2}\left(L(q, \dot{q, t})\right)dt
				\]
				\[
					\tilde{L} = L + \frac{d}{dt}f(q, t)
				\]
				\[
					\tilde{S} = \int\limits_{t_1}^{t_2} \left(L + \frac{d}{dt}f(q, t)\right) =
				\]
				\[
					= \int\limits_{t_1}^{t_2} Ldt + \delta f(q, t) |_{t_1}^{t_2}
				\]
				\[
					\delta f(q, t) |_{t_1}^{t_2} = 0
				\]
				Докажем это:
				\[
					\delta f |_{t_1}^{t_2} = \frac{\D f}{\D q}\delta q|_{t_1}^{t_2} = 0
				\]
				Тогда,
				\[
					\delta\tilde{S} = \delta S = 0
				\]
				Так как
				\[
					\delta S  \T{ -- уравнение движения и } \delta f(q, t) |_{t_1}^{t_2} = 0 
				\]
				Запишем уравнение Лагранжа
				\[
					\frac{d}{dt}\frac{\D L}{\D \dot{q}} = \frac{\D L}{\D q} //(1) 
				\]
				\[
					\frac{d}{dt}\frac{\D \tilde{L}}{\D \dot{q}} = \frac{\D \tilde{L}}{\D q} //(2)
				\]
				\[
					\frac{d}{dt}\frac{\D(\tilde{L} - L)}{\D \dot{q}} = \frac{\D(\tilde{L} - L)}{\D q} //   (2) - (1)
				\]
				\[
					\frac{d L}{dt} = \frac{\D f}{\D t} + \frac{\D L}{\D q}\dot{q}
				\]
			\subsubsection{Аддитивность функции Лагранжа}
				Допустим существует множества A и B. Каждое из них образует некую механическую систему. Тогда если множество А никак не влияет на множество В и множество В никак не влияет на множества, то будет верно следующее равенство:
				\[
					L_{AB} = L(q_A, q_B, \dot{q}_A, \dot{q}_B, t) = L_A(q_A, \dot{q}_A, t) + L_B(q_B, \dot{q}_B, t)
				\]
				\[
					L_{AB} = L_A + L_B
				\]
				Запишем уравнение Лагранжа:
				\[
					\frac{d}{dt}\frac{\D L_{AB}}{\D \dot{q}_A} = \frac{\D L_{AB}}{\D q_A}
				\]
				и
				\[
					\frac{d}{dt}\frac{\D L_{AB}}{\D \dot{q}_B} = \frac{\D L_{AB}}{\D q_B} 
				\]
				\[
					\frac{\D L_{AB}}{\D \dot{q}_A} = \frac{\D L_A}{\D \dot{q}_A} + \frac{\D L_B}{\D \dot{q}_B} = \frac{\D L_A}{\D \dot{q}_A}	
				\]
				\[
					\frac{\D L_{AB}}{\D\dot{q}_B} = \frac{\D L_A}{\D\dot{q}_B} + \frac{\D L_B}{\D\dot{q}_B} = \frac{\D L_B}{\D\dot{q}_B}
				\]
	\section{Принцип относительности Галилея}
		$\exists$системы отсчёта, в которых пространство однородно и изотропно, а время --- однородно. \\
		Такие системы отсчёта называются инерциальными\\
		Постулаты:
			\begin{enumerate}
				\item Однородное пространство --- пространство, в котором все точки пространства равноправны 
				\item Изотропность пространства --- повороты пространства не меняют законов
				\item Пространство с однородностью времени --- пространство, в котором все моменты времени равноправны
			\end{enumerate}
		Рассмотрим материальную точку.\\
		Для неё существует соответсвующая функция Лагранжа $L(q. \dot{q}, t)$ или $L(\vec{r}, \vec{v}, t)$\\
		Тогда зависимость от времени пропадает из-за однородности времени, зависимость от координаты пропадает из-за однородности пространства, а  зависимость от $\vec{v}$ преобразуется в зависимость от $|\vec{v}|$\\
		Получаем $L(\vec{r}, \vec{v}, t) \to L(|\vec{v}|)$\\
		Тогда давайте рассматривать зависимость не от $|\vec{v}|$, а от $v^2$\\
		Таким образом, в случае материальной точки функция Лагранжа зависит только от модуля скорости
		\[
			\frac{d}{dt}\frac{\D L}{\D \vec{v}} = \frac{\D L}{\D \vec{r}} = (\frac{\D L}{\D r_1}, \dots, \frac{\D L}{\D r_n})
		\]
		Для материальной точки:
		\[
			\frac{d}{dt}\frac{\D L(v^2)}{\D \vec{v}} = \frac{\D L(v^2)}{\D \vec{r}} = 0 // \T{ т.к. L не зависит от } \vec{r} 
		\]
		Рассмотрим $\vec{A}$ такой, что 
		\[
			A_i = \frac{\D L(v^2)}{\D v_i} \Indexes
		\]
		\[
			A_i = \frac{dL(v^2)}{d(v^2)}\frac{\D v^2}{\D v_i} \Indexes
		\]
		\[
			A_i = 2v_iL'(v^2) 
		\]
		Материальная точка в инерциальной системе отсчёта движется с постоянной скоростью\\
		
		Допустим, что существуют две системы отсчёта $K$ и $K'$. При этом $K$ движется относительно $K'$ со скоростью $\vec{\vee}$\\
		ВАЖНО: $|\vee| < < c$\\
		Тогда преобразования имеют вид:
		\[
			\vec{r} = \vec{r'} + \vec{\vee}t
		\] 
		\[
			t = t'
		\]
		\[
			\frac{d\vec{r}}{dt} = \frac{d\vec{r'}}{dt} + \vec{vee}
		\]
		\[
			\vec{v} = \vec{v'} + \vec{\vee}
		\]
		Постулат принципа отностиельности: законы механики одинаковы во всех инерциальных системах отсчёта\\
		Инвариантность --- величина не меняется\\
		Ковариантность --- функция не меняется(например законы механики)\\
 	\section{Функция Лангранжа свободной материальной точки}
	 	\subsection{Материальная точка в различных системах координат}
		 	Допустим у нас есть материальная точка, движущаяся со скоростью $\vec{v}$. Рассмотрим её движение в системе отсчёта, двигающуюся относительно инерциальной со скоростью $\vec{\varepsilon} << \vec{v}$
		 	Тогда
		 	\[
			 	\vec{v} = \tilde{\vec{v}} + \vec{\varepsilon}
		 	\] 
		 	\[
			 	\tilde{L} = L((\vec{v} + \vec{\varepsilon})^2) = L(v^2 + 2\vec{v}\vec{\varepsilon} + \underbrace{\varepsilon^2}_{\to 0})
		 	\]
		 	\[
			 	\tilde{L} = L(v^2) + \underbrace{L'(v^2)2\vec{v}\vec{\varepsilon}}_{\frac{d}{dt}L \exists} =
		 	\]
		 	\[
				= L(v^2) + 2\frac{d}{dt}(L(v^2)\vec{r}\vec{\varepsilon})
		 	\]
		 \subsection{Масса}
		 	Таким образом 
		 	\[
			 	L(v^2) = Av^2 + B //\T{ мы можем подобрать такую скорость, чтобы B = 0}
		 	\]
		 	Пусть $A = \frac{m}{2}$\\
		 	Тогда 
		 	\[
			 	L(v^2) = \frac{mv^2}{2}
		 	\]
		 	\[
			 	\tilde{L} = \frac{m\tilde{v}^2}{2} = 
			\] 	
			\[
			 	 = \frac{m}{2}(\vec{v} + \vec{\vee})^2 =
			\]
			\[
			 	 = \frac{mv^2}{2} + m\vec{v}\vec{\vee} + \frac{m\vee^2}{2} =
		 	\]
		 	\[
			 	= \frac{mv^2}{2} + m\frac{d\vec{r}}{dt}\vec{\vee} + \frac{d}{dt}(\vee^2t) =
		 	\]
		 	\[
			 	= \frac{mv^2}{2} + \frac{d}{dt}(m\vec{r}\vec{\vee} + \frac{m}{2}\vee^2t)
		 	\]
		 	Получаем, что при больших $\vee$ уравнение изменяет свой вид
		 	\[
				L = \sum\limits_{i = 1}^{n}\frac{m_i v_i}{2}
		 	\]
		 	ВАЖНО: $m > 0$\\
		 	Пусть возможна отрицательная масса.\\
		 	Тогда 
		 	\[
			 	S = \int\limits_{t_1}^{t_2}\left(\frac{mv^2}{2}\right)dt
		 	\]
		 	Тогда при существовании объектов с отрицательной массой мы не имеем минимума действия и это плохо(означает, что мы можем двигаться с большой скоростью и чем длиннее путь и больше скорость, тем выгоднее действие, что неправда)\\
		 	
	 	\subsection{Потенциальная энергия}
		 	\[
			 	L = \underbrace{\sum\limits_{a = 1}^{n}\frac{m_a v_a^2}{2}}_{\T{Кинетическая энергия(T)}} - \underbrace{U(\vec{r}_1, \dots, \vec{r}_n)}_{\T{Потенциальная энергия(U)}}
		 	\]
		 	\[
			 	\frac{d}{dt}\frac{\D L}{\D \dot{x}_{a_i}} = \frac{\D L}{\D x_{a_i}}
		 	\]
		 	\[
			 	\vec{r}_a = (x_{a_1}, \dots, x_{a_n})
			\]
			\[
				v^2_a = v^2_{a_1} + \dots + v^2_{a_n}
			\]
			\[
				\frac{\D L}{\D \dot{x}_{a_i}} = \dot{x}_{a_i}\frac{m}{2} = m_a x_{a_i}
			\]
			\[
				\frac{d}{dt}(m_a\dot{x}_{a_i}) = m_a\ddot{x}_{a_i}
			\]
			\[
				m_a\ddot{x_{a_i}} = \left(\frac{\D L}{\D x_{a_i}} = -\frac{\D U}{\D x_{a_i}}\right)
			\]
			\[
				m_a\ddot{x_{a_i}} = F_{a_i}
			\]
			\[
				x_{a_i} = f_{a_i}(q_1, \dots, q_s)
			\]
			\[
				v_{a_i} = \dot{x}_{a_i} = \sum\limits_{\kappa = 1}^s\frac{\D f_{a_i}}{q_\kappa}\dot{q}_\kappa
			\]
		\subsection{Общий вид кинетической энергии}
			\[
				T = \sum\limits_{a = 1}^n\sum\limits_{i = 1}^p\frac{m_a v^2_{a_i}}{2} =// \Where p \T{ --- мерность}
			\]
			\[
				= \sum\limits_{a = 1}^n\frac{m_a}{2}\sum\limits_{i = 1}^p\left(\sum\limits_{\alpha = 1}^{s}\frac{\D f_{a_i}}{\D q_\alpha}\dot{q}_\alpha\right)\times\left(\sum\limits_{\beta = 1}^s\frac{\D f_{a_i}}{\D q_\beta}\dot{q}_\beta\right) =
			\]
			\[
				= \frac{1}{2}\sum\limits_{\alpha, \beta = 1}^s\left[\sum\limits_{a = 1}^n m_a \sum\limits_{i = 1}^p \frac{\D f_{a_i}}{\D q_\alpha}\frac{\D f_{a_i}}{\D q_b} \right]\dot{q}_\alpha\dot{q}_\beta
			\]
			\[
				T = \frac{1}{2}\sum\limits_{\alpha, \beta = 1}^s\Xi_{\alpha\beta}(q)\dot{q}_\alpha\dot{q}_\beta
			\]
			\[
				L = \frac{1}{2}\sum\limits_{\alpha, \beta = 1}^{s}\Xi_{\alpha\beta}(q)q_\alpha q_\beta - U(q_1, \dots, q_n)
			\]
			\[
				\frac{d}{dt}\frac{\D L}{\D \dot{q}_\gamma} = \frac{\D L}{\D q_\gamma}, \gamma \in (1, \dots, s)
			\]
\end{document}
