\documentclass[a4paper, 12pt, titlepage]{article}
\usepackage[cm]{fullpage}
\usepackage[T2A]{fontenc}
\usepackage[utf8]{inputenc}
\usepackage[russian]{babel}
\usepackage{times}
\usepackage{color}
\usepackage{xcolor}
\usepackage[pdftex]{graphicx}
\usepackage{indentfirst}
\usepackage{listings}
\usepackage{amsfonts}

\newtheorem{Def}{Определение}[section]
\newtheorem{Th}{Теорема}

\newcommand{\Real}{\mathbb{R}}
\newcommand{\Int}{\mathbb{Z}}
\newcommand{\Nat}{\mathbb{N}}
\newcommand{\Rat}{\mathbb{Q}}
\newcommand{\Com}{\mathbb{C}}
\newcommand{\Irr}{\mathbb{I}}

\begin{document}
	\begin{center}
		\begin{Large}
			Топологические пространства
		\end{Large}
	\end{center}
	\section{Определения}
		\subsection{Топология}
			$M_0$ -- множество
			\begin{Def}
				Топологией в $M_0$ называется $\forall$ система $\tau$ его подмножеств
			\end{Def}
		\subsection{Аксиомы топологии}
			\begin{enumerate}
				\item $M_0$ и 
			\end{enumerate}
	\section{Классификация точек и подможеств}
	\section{Базы и предбазы. Аксиомы счётности}
\end{document}